\documentclass{article}
\usepackage{graphicx}
\usepackage[utf8]{inputenc}

\begin{document}

\title{Documentation: 'Simple' Interpreter in Python}
\author{Philipp Kochanski, Joshua Schmidt}

\maketitle

\begin{abstract}
The purpose of this project was to extend the 'Simple' Programming Language.
During the semester we have build our own implementation of the object model and interpreter, which were both based
on the previously provided 'Simple' parser, lexer and AST. The below described extension provides support for additional data types (boolean, float, string, list, map/dict), as well as related built-in functions/operations that are commonly available when working with these types. As a starting point for this project we have decided to use our own implementation of the interpreter, instead of the one that was provided to us at the end of the course exercises. 
\end{abstract}

\section{Changes}

\section{New Data Types}
\subsection{Integer/Float}
The new 'W\_Float' type follows the previously available 'W\_Integer' definition. It uses all 'W\_Integer' supported arithmetic operations (built-ins) and allows for operations with mixed typed operands eg. it is possible to add an integer and a float without explicit type convertion of operands. Note that the result of arithmetic operations with mixed typed operands will always be of type 'W\_Float'. In a scenario where explicit type convertion is needed, we have also included functions that parse integer to float, integer to string, floats to string and vice versa. Additionals function for example comparistion operations can be found in the 'Built-ins' section [\ref{intfloat}].

\subsection{Boolean}
In order to write more complex programs with combined logic constraints for example within if statements we implemented the 'W\_Bool' object extending 'W\_NormalObject'.
This type has several built-in methods to combine boolean expressions like and, or, not, nor, nand, xor and xnor.
The semantics are like \texttt{b1 and(b2)} where b2 can be another nested method returning a boolean value. Examples can be found in the "Built-ins" section below.


\subsection{String}
The string object 'W\_String' extends 'W\_NormalObject' and is True if the string's value is not empty.
This type provides some additional built-in methods to be able to combine or compare several strings.

\subsection{List}

\subsection{Dictionary}
\label{dicts}
The dictionary object 'W\_Dict' extends 'W\_NormalObject' and is internally implemented using a Python dictionary.
Relating to the semantics of a dictionary we decided to represent each element as a key-value tuple, the same way it is done in Python, i.e. using the form \texttt{key:value} for its elements. In addition, lists and dictionaries may not be used as keys. Therefore only 'W\_String', 'W\_Integer' and 'W\_Float' objects are considered valid keys for a dictionary entry (as it is in Python). Furthermore we have decided to allow mixed types of keys in a single dictionary. Implemented built-ins can be found in the 'Built-ins' section [\ref{builtin_dict}] below.

\subsection{Key-Value Tuple}
Since we have decided to implement dictionaries like it is done in Python, i.e. using the semantic \texttt{key:value} for the elements, we need to provide such an object.
Essentially a W\_KeyValue object is just a wrapper for two values semantically splited by a colon.
Like described in [\ref{dicts}] valid keys are of the type 'W\_String', 'W\_Integer' or 'W\_Float'. This object does not provide any built-in methods because it is just used within dictionaries.

\section{Built-ins}
In addition to the above mentioned data types, we have also added several built-in functions to enrich available functionality. Built-ins are either directly written using Python (see interpreter code - eval\_PrimitiveMethodCall) or by using already available "Simple" built-in functions and constructs. We have also provided examples as well as a test file (test\_doc\_examples.py) that demonstrates how built-ins are used.

\subsection{Integer/Float}
\label{intfloat}
	The result of arithmetic operations where mixed numeric types are used eg. int and float, will always be of type float.

\begin{itemize}
\item add(x): Add a number x (int/float)

Example: \textit{sum = 5 add(5)}

\item sub(x): Substract a number x (int/float)

Example: \textit{dif = 5 sub(5)}

\item mul(x): Multiply with number x (int/float)

Example: \textit{prod = 5 mul(5)}

\item div(x): Divide with number x (int/float)

Example: \textit{quot = 5 div(5)}

\item equals(x): '$=$' operator (for int/float) - returns True/False

Example: \textit{s = 5 equals(5)}

\item less\_than(x): '$<$' operator (for int/float) - returns True/False

Example: \textit{s = 5 less\_than(6)}

\item less\_equal(x): '$<=$' operator (for int/float) - returns True/False

Example: \textit{s = 5 less\_equal(5)}

\item greater\_than(x): '$>$' operator (for int/float) - returns True/False

Example: \textit{s = 6 greater\_than(5)}

\item greater\_equal(x): '$>=$' operator (for int/float) - returns True/False

Example: \textit{s = 5 greater\_equal(5)}
\end{itemize}


\subsection{String}
\begin{itemize}
\item len: Returns length of string

Example: \textit{length = '01234' len}
\item append(x): Append string x to string (appending a number does not convert number to string implicitly)

Example: \textit{appended = '01234' append('56789')}
\item reverse: Returns reversed version of string - keeps original string untouched

Example: \textit{reversed = '0123456789' reverse}
\item equals(x): Checks whether two strings are equal - returns boolean (True/False)

Example: \textit{eq = '0123456789' equals('9876543210')}
\item substring(x,y): Returns substring starting at x and ending before y

Example: \textit{sub = "substring" substring(0,3)}
\end{itemize}

\subsection{Boolean}
\begin{itemize}
\item not: If the left side is False, then True, else False

Example: \textit{b = True not}

\item and(x): If the left side is False, then False, else x

Example: \textit{b = True and(False)}

\item or(x): If the left side is False, then x, else True

Example: \textit{b = True or(False)}

\item nand(x): negation of 'and' operation

Example: \textit{b = True nand(False)}

\item nor(x): negation of 'or' operation

Example: \textit{b = True nor(False)}

\item xor(x): exclusive or - True if one side only is true - otherwise False

Example: \textit{b = True xor(True)}

\item xnor(x): exclusive 'nor' operation

Example: \textit{b = True xnor(False)}

\item impl(x): Implication '$\Rightarrow$' 

Example: \textit{b = True impl(True)}

\item equals(x): Simple comparison whether left sind and x are equal

Example: \textit{b = True equals(True)}
\end{itemize}


\subsection{List}
\begin{itemize}
\item add(x): Add element x to (end of) list

Example: \\ \textit{l = [1,2,3]} \\ \textit{l add(4)}

\item insert(x,y): Insert element y to list at position x - moves existing elements of list one position to the right. If position does not exisit, new element is simply added to the end of the list.

Example: \\ \textit{l = [1,3]} \\ \textit{l insert(1,2)}

\item replace(x,y): Replace element at position x with element y. If posiition does not exist, new element is simply added to the end of the list.

Example: \\ \textit{l = [1,3]} \\ \textit{l replace(1,2)}

\item del(x): Remove element at pos x from list

Example: \\ \textit{l = [1]} \\ \textit{l del(0)}

\item get(x): Get element at pos x from list

Example: \textit{e = [1,2,3] get(0)}

\item len: Get length of list

Example: \textit{length = [1,2,3] len}

\item reverse: Returns reversed version of list - keeps original list untouched

Example: \textit{r = [1,2,3] reverse}

\item oreverse: Returns reversed version of list - original list is modified

Example: \textit{r = [1,2,3] oreverse}

\item extend(x): Extends list by adding elements from x to left side (basically merge of two lists)

Example: \textit{l = [1,2,3] extend([4,5,6])}

\item clear: Clears list/Removes all stored elements

Example: \\ \textit{l = [1,2,3]} \\ \textit{l clear}
\end{itemize}

\subsection{Dictionary}
\label{builtin_dict}
\subsection{Mixed}
\begin{itemize}
\item to\_int(x): Convert given float or string (if possible) to integer

Example: \textit{i = to\_int(2.5)}

\item to\_float(x): Convert given integer or string (if possible) to float

Example: \textit{f = to\_float("2.5")}

\item to\_str(x): Convert given integer or float to string

Example: \textit{s = to\_str(1.337)}

\item ceil(x): Returns the ceiling of x, the smallest integer greater than or equal to x.

Example: \textit{ce = ceil(2.6)}

\item floor(x): Returns the floor of x, the largest integer less than or equal to x.

Example: \textit{fl = floor(2.6)}

\item s\_range(x): Simple range - returns a list where first element is 0 and last element is x

Example: \textit{l = s\_range(5)}

\item e\_range(x,y): Extended range - returns a list where first element is x and last element is y

Example: \textit{l = e\_range(5,10)}

\item fibonacci(x): Returns the x-th number from the Fibonacci sequence

Example: \textit{fib = fibonacci(8)}
\end{itemize}



\section{Conclusion}
The mentioned extensions to the interpreter were implemented to provide the, in our opinion, most important language features that should be available before implementing additional extensions like a graphical user interface, coroutines or network support. We consider the current state of the interpreter as quite mature. The provided features make it possible to write reasonable programs using the 'Simple' programming language.


\newpage

\end{document}