
\documentclass{beamer}
\usepackage[utf8]{inputenc}
\begin{document}
\title{Lua}   
\date{1. Juni 2016} 

\frame{\titlepage
        \includegraphics[scale=0.5]{img/lua_logo.png}} 

\frame{\frametitle{Lua} 
\begin{itemize}
    \item imperative Skriptsprache
        \begin{itemize}
            \item implementiert in Ansi-C
            \item unterstützt funktionale und objektorientierte Programmierung
        \end{itemize}
    \item plattformunabhängig, wird in Bytecode übersetzt
    \item sehr kleiner Interpreter (120KB), hohe Geschwindigkeit
    \item LuaJIT
    \item vorrangige Verwendung als eingebettete Sprache 
\end{itemize}
}

\frame{\frametitle{Syntax-Besonderheiten} 
\begin{itemize}
    \item Datentypen
        \begin{itemize}
            \item nil, boolean, number, string, function, userdata und thread
            \item einziger strukturierter Datentyp: table
            \item eine Tabelle ist eine Menge von Key-Value Paaren
            \item die Indizierung beginnt bei 1
        \end{itemize}
\end{itemize}
}

\frame{\frametitle{Syntax-Besonderheiten} 
\begin{itemize}
    \item Variablen
        \begin{itemize}
            \item sind nicht typgebunden, dynamische Zuweisung
            \item globale Definition, außer mit Schlüsselwort \textbf{local}
        \end{itemize}
    \item Funktionen
        \begin{itemize}
            \item Schlüsselwort \textbf{function}
            \item First-Class-Objekt
            \item können während der Laufzeit dynamisch erzeugt und verändert werden
        \end{itemize}
\end{itemize}
}

\frame{\frametitle{Ist Lua eine dynamische Sprache?} 
\begin{itemize}
    \item dynamische Typisierung \hfill\includegraphics[scale=0.5]{img/correct.jpg}
    \item Garbage Collection  \hfill\includegraphics[scale=0.5]{img/correct.jpg}
    \item Objektorientierung \hfill\includegraphics[scale=0.5]{img/correct.jpg}
    \item interaktiv \hfill\includegraphics[scale=0.5]{img/correct.jpg}
    \item Reflexion/Introspektion \hfill\includegraphics[scale=0.5]{img/correct.jpg}
    \item interpretiert \hfill\includegraphics[scale=0.5]{img/correct.jpg}
    \item Latebound Everything \hfill\includegraphics[scale=0.5]{img/correct.jpg}
\end{itemize}
}
\end{document}

